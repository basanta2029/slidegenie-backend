
\documentclass[11pt,16:9,handout]{beamer}

% Theme configuration
\usetheme{Madrid}
\usecolortheme{dolphin}
\usefonttheme{default}



% Navigation symbols
\setbeamertemplate{navigation symbols}{}

% Package imports
\usepackage[utf8]{inputenc}
\usepackage[T1]{fontenc}
\usepackage[english]{babel}
\usepackage{amsmath}
\usepackage{amsfonts}
\usepackage{amssymb}
\usepackage{graphicx}
\usepackage{booktabs}
\usepackage{array}
\usepackage{longtable}
\usepackage{multirow}
\usepackage{multicol}
\usepackage{xcolor}
\usepackage{tikz}
\usepackage{pgfplots}
\pgfplotsset{compat=1.18}
\usepackage[colorlinks=true,linkcolor=blue,citecolor=red,urlcolor=blue]{hyperref}
\usepackage{url}
\usepackage{subcaption}

% Bibliography configuration
\usepackage[backend=biber,style=authoryear-comp,sorting=nyt,maxnames=3,minnames=1]{biblatex}
\addbibresource{references.bib}

% TikZ libraries


% Custom commands
\newcommand{\CNN}{\text{CNN}}
\newcommand{\RNN}{\text{RNN}}
\newcommand{\LSTM}{\text{LSTM}}
\DeclareMathOperator{\softmax}{softmax}

% Title page information
\title{Deep Learning for Computer Vision:\textbackslash\{\}\textbackslash\{\}Recent Advances and Applications}
\author{Dr. Elena Rodriguez\textbackslash\{\}\textbackslash\{\}Joint work with Alex Kim and Maria Santos}
\institute{AI Research Lab\\Institute of Technology}
\date{Conference on AI 2024}

% Document settings
\setbeamertemplate{frametitle continuation}{(\insertcontinuationcount)}
\setbeamertemplate{section in toc}[sections numbered]
\setbeamertemplate{subsection in toc}[subsections numbered]

% Handout configuration
\usepackage{pgfpages}
\pgfpagesuselayout{4 on 1}[a4paper,border shrink=2mm,landscape]
\setbeameroption{show notes}

\begin{document}

% Title frame
\begin{frame}
    \titlepage
\end{frame}

% Table of contents

\begin{frame}{Outline}
    \tableofcontents
\end{frame}



\begin{frame}[fragile]{Introduction}

            Deep learning has revolutionized computer vision \cite{lecun2015deep}:
            
\begin{itemize}
  \item \textbf{Image Classification}: Surpassed human performance on ImageNet
  \item \textbf{Object Detection}: Real-time detection in autonomous vehicles
  \item \textbf{Semantic Segmentation}: Pixel-level understanding
  \item \textbf{Generative Models}: Creating realistic synthetic images
\end{itemize}
            
            The transformer architecture \cite{attention2017} has also found applications in vision tasks.
            

\end{frame}



\begin{frame}{Network Architecture}
Our proposed architecture combines multiple components:


\begin{figure}[center]
    \centering
    \includegraphics[width=0.95\textwidth]{network_architecture.png}
    \caption{Overview of the proposed deep network architecture}
    \label{fig:architecture}
\end{figure}


\begin{equation}
\text{Output} = \softmax(W_3 \cdot \text{ReLU}(W_2 \cdot \text{ReLU}(W_1 \cdot \mathbf{x} + \mathbf{b}_1) + \mathbf{b}_2) + \mathbf{b}_3)
\end{equation}

\end{frame}



\begin{frame}{Experimental Results}
Performance comparison on standard benchmarks:


\begin{table}[center]
    \centering
    \caption{Accuracy (\textbackslash\{\}\%) comparison on image classification benchmarks}
    \begin{tabular}{lllll}
    \toprule
        Method & CIFAR-10 & CIFAR-100 & ImageNet & Parameters \\
        \midrule
        ResNet-50 & 95.3 & 78.1 & 76.2 & 25.6M \\
        DenseNet-121 & 95.1 & 77.9 & 74.4 & 8.0M \\
        EfficientNet-B0 & 96.7 & 81.2 & 77.3 & 5.3M \\
        Our Method & **97.2** & **82.8** & **78.9** & 7.1M \\
        \bottomrule
    \end{tabular}
    
    \label{tab:results}
\end{table}


\begin{equation}
\text{Improvement} = \frac{\text{Our Accuracy} - \text{Baseline Accuracy}}{\text{Baseline Accuracy}} \times 100\%
\end{equation}

\end{frame}



\begin{frame}{Ablation Study}
Analysis of individual component contributions:


\begin{figure}[center]
    \centering
    
    \begin{subfigure}[b]{0.48\textwidth}
        \centering
        \includegraphics[width=\textwidth]{ablation_accuracy.png}
        \caption{Accuracy vs. components}
        
    \end{subfigure}

    \begin{subfigure}[b]{0.48\textwidth}
        \centering
        \includegraphics[width=\textwidth]{ablation_loss.png}
        \caption{Training loss curves}
        
    \end{subfigure}
    \caption{Ablation study results}
    \label{fig:ablation}
\end{figure}


\end{frame}



\begin{frame}[allowframebreaks]{Conclusions and Future Work}

            \textbf{Key Contributions:}
\begin{itemize}
  \item Novel architecture achieving SOTA results
  \item Efficient design with fewer parameters
  \item Comprehensive evaluation on multiple datasets
\end{itemize}
            
            \textbf{Future Directions:}
\begin{itemize}
  \item Extension to video understanding
  \item Multi-modal learning applications
  \item Deployment optimization for mobile devices
\end{itemize}
            
            \textbf{Acknowledgments:} This work was supported by NSF Grant AI-2024-001.
            

\end{frame}


% Bibliography

\begin{frame}[allowframebreaks]{References}
    \printbibliography
\end{frame}


\end{document}

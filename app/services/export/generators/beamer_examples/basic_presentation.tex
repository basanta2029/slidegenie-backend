
\documentclass[11pt,16:9,handout]{beamer}

% Theme configuration
\usetheme{Berlin}
\usecolortheme{default}
\usefonttheme{default}



% Navigation symbols
\setbeamertemplate{navigation symbols}{}

% Package imports
\usepackage[utf8]{inputenc}
\usepackage[T1]{fontenc}
\usepackage[english]{babel}
\usepackage{amsmath}
\usepackage{amsfonts}
\usepackage{amssymb}
\usepackage{graphicx}
\usepackage{booktabs}
\usepackage{array}
\usepackage{longtable}
\usepackage{multirow}
\usepackage{multicol}
\usepackage{xcolor}
\usepackage{tikz}
\usepackage{pgfplots}
\pgfplotsset{compat=1.18}
\usepackage[colorlinks=true,linkcolor=blue,citecolor=red,urlcolor=blue]{hyperref}
\usepackage{url}
\usepackage{subcaption}

% Bibliography configuration


% TikZ libraries


% Custom commands


% Title page information
\title{Introduction to Machine Learning}
\author{Dr. Sarah Johnson}
\institute{Computer Science Department\\University of Technology}
\date{\today}

% Document settings
\setbeamertemplate{frametitle continuation}{(\insertcontinuationcount)}
\setbeamertemplate{section in toc}[sections numbered]
\setbeamertemplate{subsection in toc}[subsections numbered]

% Handout configuration
\usepackage{pgfpages}
\pgfpagesuselayout{4 on 1}[a4paper,border shrink=2mm,landscape]
\setbeameroption{show notes}

\begin{document}

% Title frame
\begin{frame}
    \titlepage
\end{frame}

% Table of contents

\begin{frame}{Outline}
    \tableofcontents
\end{frame}



\begin{frame}{What is Machine Learning?}

            Machine Learning is a branch of artificial intelligence that focuses on:
            
\begin{itemize}
  \item \textbf{Supervised Learning}: Learning from labeled data
  \item \textbf{Unsupervised Learning}: Finding patterns in unlabeled data
  \item \textbf{Reinforcement Learning}: Learning through interaction
\end{itemize}
            
            > \textit{"Machine learning is the field that gives computers the ability to learn without being explicitly programmed."}
            > — Arthur Samuel (1959)
            

\end{frame}



\begin{frame}{Types of Machine Learning}

            We can categorize ML algorithms into several types:
            
\begin{enumerate}
  \item \textbf{Supervised Learning}
\begin{itemize}
\end{enumerate}
  \item Classification (discrete output)
  \item Regression (continuous output)
\end{itemize}
            
\begin{enumerate}
  \item \textbf{Unsupervised Learning}
\begin{itemize}
\end{enumerate}
  \item Clustering
  \item Dimensionality reduction
  \item Association rules
\end{itemize}
            
\begin{enumerate}
  \item \textbf{Reinforcement Learning}
\begin{itemize}
\end{enumerate}
  \item Policy-based methods
  \item Value-based methods
\end{itemize}
            

\[ \text{Classification: } f: X \rightarrow \{1, 2, \ldots, K\} \]

\[ \text{Regression: } f: X \rightarrow \mathbb{R} \]

\end{frame}



\begin{frame}{Performance Evaluation}
Common metrics for evaluating ML models:


\begin{table}[center]
    \centering
    \caption{Evaluation metrics by problem type}
    \begin{tabular}{lll}
    \toprule
        Metric & Classification & Regression \\
        \midrule
        Accuracy & ✓ & ✗ \\
        Precision/Recall & ✓ & ✗ \\
        F1-Score & ✓ & ✗ \\
        MSE/RMSE & ✗ & ✓ \\
        MAE & ✗ & ✓ \\
        R² & ✗ & ✓ \\
        \bottomrule
    \end{tabular}
    
    \label{tab:metrics}
\end{table}


\begin{equation}
\text{Accuracy} = \frac{\text{Correct Predictions}}{\text{Total Predictions}}
\end{equation}

\begin{equation}
\text{MSE} = \frac{1}{n}\sum_{i=1}^{n}(y_i - \hat{y}_i)^2
\end{equation}

\end{frame}


% Bibliography


\end{document}
